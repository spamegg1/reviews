\documentclass[14pt]{extarticle} 
\usepackage{amsmath,mathtools,amsfonts,amsthm,amssymb,hyperref}
\usepackage{wasysym,geometry,bussproofs,latexsym,parskip,bookmark}
\newtheorem{defn}{Definition}
\newtheorem{thm}{Theorem}
\newtheorem{claim}{Claim}
\newtheorem{lemma}{Lemma}
\hypersetup{colorlinks,allcolors=blue,linktoc=all}
\geometry{a4paper} 
\geometry{margin=0.5in}
\title{Math for CS 2015/2019 Problem Set 8 solutions}
\author{https://github.com/spamegg1}
\begin{document}
\maketitle
\tableofcontents

\section{Problem 1}
Prove Corollary 11.10.12: If all edges in a finite weighted graph have distinct weights, then the graph has a unique MST in the course textbook.

{\it Hint:} Suppose $M$ and $N$ were different MST’s of the same graph. Let $e$ be the smallest edge in one and not the other, say $e \in M - N$, and observe that $N + e$ must have a cycle.
\begin{proof}
First let's clarify: when we are talking about MST's, the graph $G$ is always finite, connected, edge-weighted and undirected, and the edge weights are always positive (even though the textbook does not make this clear due to its informal conversational style of writing).

Let's follow the hint. Recall the notation $w(e)$ to denote the edge weight of an edge $e$.

1. Assume $G$ is a finite weighted graph with all edges having distinct weights.

2. Argue by contradiction and assume $G$ has two different MST's, $M$ and $N$. Notice that $w(M) = w(N)$, since they are both MST's of the same graph $G$.

3. Since $M$ and $N$ are different, there exists an edge that belongs to one of them but not both. So $(M-N) \cup (N-M)$ is nonempty. 

4. Let $e$ be the edge in $(M-N) \cup (N-M)$ with the smallest weight. We don't know if $e \in M-N$ or $e \in N-M$, but these two cases have similar proofs, so without loss of generality, assume $e \in M-N$.

5. Say $e$ is the edge between the two nodes $a$ and $b$. Since $G$ is connected, there is a path in $G$ between the nodes $a$ and $b$. 

6. Since $N$ spans $G$, $N$ contains edges that touch all the nodes on this path. Since $N$ is a tree, there is a path $p$ in $N$ between the nodes $a$ and $b$.

7. Since $e$ is not in $N$, $e$ is not part of the path $p$. So $N + e$ contains a cycle that starts and ends at $a$. (The path $p$ starts at $a$ and ends at $b$, then $e$ connects $b$ to $a$, completing the cycle.)

8. Case 1: all the edges of the path $p$ are in $M \cap N$.

?. Case 2: there exists an edge $g$ on the path $p$ that belongs to $(M-N) \cup (N-M)$.
\end{proof}

\section{Problem 2}
A basic example of a simple graph with chromatic number $n$ is the complete graph on $n$ vertices, that is $\chi (K_n) = n$. This implies that any graph with $K_n$ as a subgraph must have chromatic number at least $n$. It’s a common misconception to think that, conversely, graphs with high chromatic number must contain a large complete subgraph. In this problem we exhibit a simple example countering this misconception, namely a graph with chromatic number four that contains no {\it triangle} (length three cycle) and hence no subgraph isomorphic to $K_n$ for $n \geq 3$. Namely, let $G$ be the 11-vertex graph of Figure 1. The reader can verify that $G$ is triangle-free.
\subsection{(a)}
Show that $G$ is 4-colorable.
\begin{proof}
\end{proof}
\subsection{(b)}
Prove that $G$ can't be colored with 3 colors.
\begin{proof}
\end{proof}

\section{Problem 3}
The preferences among 4 boys and 4 girls are partially specified in the following table:
$$
\begin{array}{|c c c c c|}
\hline
B1: & G1 & G2 & - & - \\
B2: & G2 & G1 & - & - \\
B3: & - & - & G4 & G3 \\
B4: & - & - & G3 & G4 \\
\hline 
G1: & B2 & B1 & - & - \\
G2: & B1 & B2 & - & - \\
G3: & - & - & B3 & B4 \\
G4: & - & - & B4 & B3 \\
\hline 
\end{array}
$$
\subsection{(a)}
Verify that
$$
(B1, G1), (B2, G2), (B3, G3), (B4, G4)
$$
will be a stable matching whatever the unspecified preferences may be.
\begin{proof}
\end{proof}
\subsection{(b)}
Explain why the stable matching above is neither boy-optimal nor boy-pessimal and so will not be an outcome of the Mating Ritual.
\begin{proof}
\end{proof}
\subsection{(c)}
Describe how to define a set of marriage preferences among $n$ boys and $n$ girls which have at least $2^{n/2}$ stable assignments.

{\it Hint:} Arrange the boys into a list of $n/2$ pairs, and likewise arrange the girls into a list of $n/2$ pairs of girls. Choose preferences so that the $k$th pair of boys ranks the $k$th pair of girls just below the previous pairs of girls, and likewise for the $k$th pair of girls. Within the $k$th pairs, make sure each boy’s first choice girl in the pair prefers the other boy in the pair.
\begin{proof}
\end{proof}
\end{document}