\documentclass[14pt]{extarticle} 
\usepackage{amsmath,mathtools,amsfonts,amsthm,amssymb,hyperref}
\usepackage{wasysym,geometry,bussproofs,latexsym,parskip,bookmark}
\newtheorem{defn}{Definition}
\newtheorem{thm}{Theorem}
\newtheorem{claim}{Claim}
\newtheorem{lemma}{Lemma}
\hypersetup{colorlinks,allcolors=blue,linktoc=all}
\geometry{a4paper} 
\geometry{margin=0.5in}
\title{Math for CS 2015/2019 Problem Set 11 solutions}
\author{https://github.com/spamegg1}
\begin{document}
\maketitle
\tableofcontents

\section{Problem 1}
\subsection{(a)}
Let R be an 82x4 rectangular matrix each of whose entries are colored red, white or blue. Explain why at least two of the 82 rows in R must have identical color patterns.
\begin{proof}
\end{proof}
\subsection{(b)}
Conclude that R contains four points with the same color that form the corners of a rectangle.
\begin{proof}
\end{proof}
\subsection{(c)}
Now show that the conclusion from part (b) holds even when R has only 19 rows.

Hint: How many ways are there to pick two positions in a row of length four and color them the same?
\begin{proof}
\end{proof}

\section{Problem 2}
We play a game with a deck of 52 regular playing cards, of which 26 are red and 26 are black. I randomly shuffle the cards and place the deck face down on a table. You have the option of “taking” or “skipping”
the top card. If you skip the top card, then that card is revealed and we continue playing with the remaining deck. If you take the top card, then the game ends; you win if the card you took was revealed to be black, and you lose if it was red. If we get to a point where there is only one card left in the deck, you must take it.

Prove that you have no better strategy than to take the top card—which means your probability of winning is 1/2.

Hint: Prove by induction the more general claim that for a randomly shuffled deck of n cards that are red or black—not necessarily with the same number of red cards and black cards—there is no better strategy
than taking the top card.
\begin{proof}
\end{proof}

\section{Problem 3}
Suppose you have three cards: Ace of Hearts, Ace of Spades, and a Jack. From these, you choose a random hand (that is, each card is equally likely to be chosen) of two cards, and let K be the number of Aces in your hand. You then randomly pick one of the cards in the hand and reveal it.
\subsection{(a)}
Describe a simple probability space (that is, outcomes and their probabilities) for this scenario, and list the outcomes in each of the following events:

1. $[K \geq 1]$, (that is, your hand has an Ace in it),

2. Ace of Hearts is in your hand,

3. the revealed card is an Ace of Hearts,

4. the revealed card is an Ace.
\begin{proof}
\end{proof}

\subsection{(b)}
Then calculate $Pr[K = 2 | E]$ for E equal to each of the four events in part (a). Notice that most, but not all, of these probabilities are equal. 
\begin{proof}
\end{proof}

Now suppose you have a deck with $d$ distinct cards, $a$ different kinds of Aces (including an Ace of Hearts), you draw a random hand with $h$ cards, and then reveal a random card from your hand.

\subsection{(c)}
Prove that Pr[Ace of Hearts is in your hand] = $h/d$.
\begin{proof}
\end{proof}

\subsection{(d)}
Prove that
$$
Pr[K = 2 \,|\, \text{Ace of Hearts is in your hand}] = Pr[K = 2] \cdot \frac{2d}{ah}
$$
\begin{proof}
\end{proof}

\subsection{(e)}
Conclude that
$$
Pr[K = 2 \,|\, \text{the revealed card is an Ace}] = Pr[K = 2 \,|\, \text{Ace of Hearts is in your hand}]
$$
\begin{proof}
\end{proof}

\end{document}