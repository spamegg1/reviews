\documentclass[14pt]{extarticle} 
\usepackage{amsmath,mathtools,amsfonts,amsthm,amssymb,hyperref}
\usepackage{wasysym,geometry,bussproofs,latexsym,parskip,bookmark}
\newtheorem{defn}{Definition}
\newtheorem{thm}{Theorem}
\newtheorem{claim}{Claim}
\newtheorem{lemma}{Lemma}
\hypersetup{colorlinks,allcolors=blue,linktoc=all}
\geometry{a4paper} 
\geometry{margin=0.5in}
\title{Math for CS 2015/2019 Problem Set 10 solutions}
\author{https://github.com/spamegg1}
\begin{document}
\maketitle
\tableofcontents

\section{Problem 1}
Suppose you have seven dice—each a different color of the rainbow, otherwise the dice are standard, with faces numbered 1 to 6. A roll is a sequence specifying a value for each die in rainbow (ROYGBIV) order.
For example, one roll is $(3, 1, 6, 1, 4, 5, 2)$ indicating that the red die showed a 3, the orange die showed 1, the yellow 6, ...

For the problems below, describe a bijection between the specified set of rolls and another set that is easily counted using the Product, Generalized Product, and similar rules. Then write a simple arithmetic formula, possibly involving factorials and binomial coefficients, for the size of the set of rolls. You do not need to prove that the correspondence between sets you describe is a bijection, and you do not need to simplify the expression you come up with.

For example, let $A$ be the set of rolls where 4 dice come up showing the same number, and the other 3 dice also come up the same, but with a different number. Let $R$ be the set of seven rainbow colors and $S \Coloneqq [1, 6]$ be the set of dice values. 

Define $B \Coloneqq P_{S,2} \times R_3$, where $P_{S,2}$ is the set of 2-permutations of $S$ and $R_3$ is the set of size-3 subsets of $R$. Then define a bijection from $A$ to $B$ by mapping a roll in $A$ to the sequence in $B$ whose first element is a pair consisting of the number that came up three times followed by the number that came up four times, and whose second element is the set of colors of the three matching dice. 

For example, the roll
$$
(4, 4, 2, 2, 4, 2, 4) \in A
$$
maps to
$$
((2, 4), \{\text{yellow}, \text{green}, \text{indigo}\}) \in B
$$
Now by the Bijection rule $|A| = |B|$, and by the Generalized Product and Subset rules,
$$
|B| = 6 \cdot 5 \cdot \binom{7}{3}
$$
\subsection{(a)}
For how many rolls do exactly two dice have the value 6 and the remaining five dice all have different values? Remember to describe a bijection and write a simple arithmetic formula.

Example: (6, 2, 6, 1, 3, 4, 5) is a roll of this type, but (1, 1, 2, 6, 3, 4, 5) and (6, 6, 1, 2, 4, 3, 4) are not.
\begin{proof}
\subsubsection{General counting principles}
I will do these problems in an unnecessarily long way, because I've run into many learners who really struggle with thinking about counting complex things.

Say we want to choose $k$ elements from a set of $n$ elements. There are 4 categories of choices with counting formulas: 

ordered without repetitions: Permutations $\displaystyle = \frac{n!}{(n-k)!}$,

ordered with repetitions: Tuples $ = n^k$,

unordered without repetitions: Combinations $\displaystyle = \binom{n}{k} = \frac{n!}{k!(n-k)!}$.

unordered with repetitions: Combinations with rep.s $\displaystyle = \binom{k+n-1}{n-1}$.

\subsubsection{A simpler version of the problem}
Let's think about a simpler, smaller problem of the same type. Imagine there were only 3 colors, only 4 roll values 1,2,3,4, and we wanted to find the rolls that have exactly one die with 4, and the other 2 had different values.

{\bf First approach}

Removing 4, the remaining values are 1,2,3. So we need 2-permutations of them: (1,2), (2,1), (1,3), (3,1), (2,3), (3,2). 

What is the formula to count this number, 6? We start with 3 elements, we want to choose 2 elements from it, without repetitions. And we want the order to matter, so that (1,2) is different than (2,1).

So, what we want is: ``ordered without repetitions''. This means permutations, with $n = 3, k = 2$. So $\frac{3!}{(3-2)!} = \frac{6}{1} = 6$.

Then we can write down all the rolls we are looking for:

For (1,2): (4,1,2), (1,4,2), (1,2,4)

For (2,1): (4,2,1), (2,4,1), (2,1,4)

For (1,3): (4,1,3), (1,4,3), (1,3,4)

For (3,1): (4,3,1), (3,4,1), (3,1,4)

For (2,3): (4,2,3), (2,4,3), (2,3,4)

For (3,2): (4,3,2), (3,4,2), (3,2,4)

How are we counting this? For each 2-permutation, say (1,2), we are inserting 4 into one of the 3 locations: (here) 1 (or here) 2 (or here). So that's simply choosing 1 thing out of 3 things = 3. That's one way to think about it. 

{\bf Second approach}

Here's another way to think: let's consider all the 2-element subsets of $\{1, 2, 3\}$, which are $\{1,2\}, \{1,3\}, \{2,3\}$. These are ``unordered without repetitions'', which is Combinations, with $n = 3, k = 2$: $\binom{3}{2} = \frac{3!}{2!(3-2)!} = \frac{6}{2} = 3$.

For each subset, say $\{1,2\}$, we add the 4 to it to create a new set $\{1,2,4\}$ and then we compute all the 3! = 6 permutations (``ordered without repetitions'') of this 3-element set: (4,1,2), (1,4,2), (1,2,4), (4,2,1), (2,4,1), (2,1,4).

\subsubsection{Generalizing the approaches}
Do these ideas work if we had more than one copy of 4 to add? Let's check. Imagine we have 4 colors, we are looking for 4 dice rolls, exactly two of which are 4, and the other 2 different.

{\bf First approach again}

In our first way of thinking, say we take the 2-permutation (1,2) of the set $\{1,2,3\}$. (Just like before there are 6 of them.) Then we insert the first copy of 4 in one of the 3 places (here) 1 (or here) 2 (or here):

(4,1,2), (1,4,2), (1,2,4).

Now we have to insert the second copy of 4 into each one of these, again in 3 places:

For (4,1,2): (4,4,1,2), (4,1,4,2), (4,1,2,4)

For (1,4,2): (4,1,4,2), (1,4,4,2), (1,4,2,4)

For (1,2,4): (4,1,2,4), (1,4,2,4), (1,2,4,4)

So there are 9 of them. But there are duplicates here: (4,1,2,4), (1,4,2,4), and (4,1,4,2) are all listed twice. Removing those we end up with 6 in total, instead of 9. So we need to figure out why the duplicates are showing up. Maybe the idea of inserting a 4 to the ``spaces in between'' is not the right idea.

Instead of inserting the two 4s one at a time, let's think of inserting both of them. The final tuple is a 4-tuple like $(1,2,4,4)$. There are 4 ``slots'' to insert two 4s, so this is a combination $\binom{4}{2} = 6$. That's the right count (which wasn't clear when we had to deal with only one 4), it gives us 6 instead of 9.

When we do the same procedure for the permutation (2,1) we get 6 more rolls that are different from the above 6. So we get a total of 12 rolls from (1,2) and (2,1). We'd get 12 more from (1,3) and (3,1), and 12 more from (2,3), (3,2) for a total of 36.

{\bf Second approach again}

From $\{1,2,3\}$ we choose the 2-element subsets (there are 3 of them as before). Then, say for the subset $\{1,2\}$ we add two 4s to it to obtain a new collection $\{4,4,1,2\}$, and we compute all the 4-permutations of this: $\frac{4!}{(4-4)!} = 24$. 

However half of these will be duplicates, because the two copies of 4 are indistinguishable. So dividing by 2! (the number of ways of ordering the 4s we have) we get 12.

We'd get 12 more from the subset $\{1,3\}$ and 12 more from $\{2,3\}$ for a total of 36 again.

\subsubsection{Counting the actual problem}
Now for the real problem. Using the first approach:

There are $\frac{5!}{(5-5)!} = 120$ different 5-permutations of the set $\{1,2,3,4,5\}$. For each permutation we would like to add two 6s to them. The final result has length 7, and we need to choose 2 spots out of 7 for placing the 6s. There are $\binom{7}{2} = 21$ ways to do that. So there are $130 \cdot 21 = 2520$ rolls of the kind we want.

Now using the second approach:

We need 7 dice rolls, exactly 2 of which is 6, and the other 5 are all different (from the set $\{1,2,3,4,5\}$). $7-2 = 5$, so we need to choose 5-element subsets of $\{1,2,3,4,5\}$, so $k = 5$. 

This is calculated with Combinations using: $\binom{n}{k} = \binom{5}{5} = \frac{5!}{5!(5-5)!} = 1$. Makes sense, there is only one 5-element subset of $\{1,2,3,4,5\}$.

Then we add two 6s to this subset to obtain the colection $\{6,6,1,2,3,4,5\}$ which has $n = 7$ elements, and find all the $k = 7$-permutations of this set: $\frac{7!}{(7-7)!} = 7! = 5040$. But there are $2! = 2$ ways to order the two 6s, so half of them are duplicated. So there are a total $5040/2 = 2520$ rolls.

\subsubsection{The bijection}

Finally let's write down the bijection. We should have done that first! Then we could have easily counted. But I included the above for the sake of learning. Writing out the bijection first is sometimes very difficult.

Define $A$ to be the set of 7 rolls where exactly 2 of the rolls are 6 and the remaining 5 are all different from each other. 

Define $S \Coloneqq [1,5]$, define $P_{S,5}$ to be the set of all 5-permutations of $S$, define $R \Coloneqq [1,7]$, define $C_{R, 2}$ to be the set of 2-combinations of $R$, and define $B = P_{S,5} \times C_{R,2}$.

Define $f: A \to B$ as follows: given a 7-roll in $A$ like 
$$
a = (6, 2, 6, 1, 3, 4, 5)
$$
define $f(a)$ to be the element $(p, c)$ of $B$ where $p$ is the ordered 5-tuple consisting of $\{1,2,3,4,5\}$ that appears in $a$ in order when we ignore the 6s in $a$ (in this case (2,1,3,4,5)), and $c$ is the pair that shows which slots are occupied by the two 6s (in this case $\{1,3\}$ because the first and third numbers in $a$ are 6s).

This bijection describes our first approach. Can you write a bijection for the second approach?
\end{proof}
\subsection{(b)}
For how many rolls do two dice have the same value and the remaining five dice all have different values? Remember to describe a bijection and write a simple arithmetic formula.

Example: (4, 2, 4, 1, 3, 6, 5) is a roll of this type, but (1, 1, 2, 6, 1, 4, 5) and (6, 6, 1, 2, 4, 3, 4) are not.
\begin{proof}
\end{proof}
\subsection{(c)}
For how many rolls do two dice have one value, two different dice have a second value, and the remaining three dice a third value? Remember to describe a bijection and write a simple arithmetic formula.

Example: (6, 1, 2, 1, 2, 6, 6) is a roll of this type, but (4, 4, 4, 4, 1, 3, 5) and (5, 5, 5, 6, 6, 1, 2) are not.
\begin{proof}
\end{proof}

\section{Problem 2}
Answer the following questions with a number or a simple formula involving factorials and binomial coefficients. Briefly explain your answers.
\subsection{(a)}
How many ways are there to order the 26 letters of the alphabet so that no two of the vowels a, e, i, o, u appear consecutively and the last letter in the ordering is not a vowel?

Hint: Every vowel appears to the left of a consonant.
\begin{proof}
\end{proof}
\subsection{(b)}
How many ways are there to order the 26 letters of the alphabet so that there are at least two consonants immediately following each vowel?
\begin{proof}
\end{proof}
\subsection{(c)}
In how many different ways can $2n$ students be paired up?
\begin{proof}
\end{proof}
\subsection{(d)}
Two $n$-digit sequences of digits 0,1,. . . ,9 are said to be of the same type if the digits of one are a permutation of the digits of the other. For $n = 8$, for example, the sequences 03088929 and 00238899 are the same type. How many types of $n$-digit sequences are there?
\begin{proof}
\end{proof}

\section{Problem 3}
\subsection{(a)}
Use the Multinomial Theorem 14.6.5 to prove that
$$
(x_1 + \ldots x_n)^p \equiv x_1^p + \ldots x_n^p \,\,\,(\text{mod}\,\,p)
$$
for all primes $p$. (Do not prove it using Fermat’s “little” Theorem. The point of this problem is to offer an independent proof of Fermat’s theorem.)

Hint: Explain why $\binom{p}{k_1, k 2, \ldots, k_n}$ is divisible by $p$ if all the $k_i$’s are positive integers less than $p$.
\begin{proof}
\end{proof}

\subsection{(b)}
Explain how (a) immediately proves Fermat’s Little Theorem 8.10.8: 
$$
n^{p-1} \equiv 1 (\text{mod }p)
$$
when $n$ is not a multiple of $p$.
\begin{proof}
\end{proof}
\end{document}