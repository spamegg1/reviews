\documentclass[14pt]{extarticle} 
\usepackage{mathtools,amssymb,parskip}
%\newcommand{\sfm}[1]{$\mathsf{#1}$}
\begin{document}
(contributed by spamegg)

{\bf Exercises 35:}

{\bf(a)} Take a QL language, with a couple of proper names, a couple of unary predicates, and a binary predicate, and consider the following q-valuation $q$:

The domain is: $\{\text{Romeo, Juliet, Benedick, Beatrice}\}$

$\mathsf{m}$: Romeo

$\mathsf{n}$: Juliet

$\mathsf{F}$: $\{\text{Romeo, Benedick}\}$

$\mathsf{G}$: $\{\text{Juliet, Beatrice}\}$

$\mathsf{L}$: $\{\langle\text{Romeo, Juliet}\rangle, \langle\text{Juliet, Romeo}\rangle, \langle\text{Benedick, Beatrice}\rangle, $

$\hspace{0.9cm}\langle\text{Beatrice, Benedick}\rangle, \langle\text{Benedick, Benedick}\rangle\}$

Then what are the truth values of the following wffs?

(1) $\mathsf{\exists xLmx}$

True. 

Informally translates to ``Romeo loves someone.''

$\langle\text{Romeo, Juliet}\rangle$ is in the extension of $\mathsf{L}$, so $\mathsf{Lmn}$ is true. Informally Juliet is an $x$ that fulfills the existential quantifier. Formally, we can expand the valuation $q$ to $q_a$ where dummy name $a$ is assigned Juliet and $\mathsf{Lma}$ is true. By the definition (Q6) $\mathsf{\exists xLmx}$ is true.

(2) $\mathsf{\forall xLxm}$

False.

Informally translates to ``everyone loves Romeo''. False since, for example, 
$$
\langle \text{Benedick, Romeo}\rangle
$$ 
is not in the extension of $\mathsf{L}$. Formally, there is an expanded valuation $q_a$, where dummy name $a$ is assigned Benedick, for which $\mathsf{Lam}$ is false. So by the definition (Q5), $\mathsf{\forall xLxm}$ is false.

I will be less formal for the rest.

(3) $(\mathsf{\exists xLmx\to Lmn})$

True. 

Both the antecedent and the consequent are true. By (1) above, the antecedent $\mathsf{\exists xLmx}$ is true. The consequent $\mathsf{Lmn}$ is also true because $\langle \text{Romeo, Juliet}\rangle$ is in the extension of $L$.

(4) $\mathsf{\forall x(Fx \to \neg Gx)}$

True.

There are 4 people in the domain, so let's verify $\mathsf{Fa \to \neg Ga}$ for all of them:

When $\mathsf{a}$ is Romeo, $\mathsf{Fa}$ is true (Romeo is in F's extension), $\mathsf{\neg Ga}$ is true (Romeo is not in G's extension), so $\mathsf{Fa \to \neg Ga}$ is true.

When $\mathsf{a}$ is Juliet, $\mathsf{Fa}$ is false (Juliet is not in F's extension), so $\mathsf{Fa \to \neg Ga}$ is true.

When $\mathsf{a}$ is Benedick, $\mathsf{Fa}$ is true (Benedick is in F's extension), $\mathsf{\neg Ga}$ is true (Benedick is not in G's extension), so $\mathsf{Fa \to \neg Ga}$ is true.

When $\mathsf{a}$ is Beatrice, $\mathsf{Fa}$ is false (Beatrice is not in F's extension), so $\mathsf{Fa \to \neg Ga}$ is true.

Therefore by (Q5) $\mathsf{\forall x(Fx \to \neg Gx)}$ is true.

(5) $\mathsf{\forall x(Gx \to (Lxm \vee \neg Lmx))}$

True.

Let's approach this like (4) but take shortcuts. We won't consider the $\mathsf{a}$'s for which $\mathsf{Ga}$ is false, because the implication $\mathsf{Ga \to \ldots}$ is automatically true in that case.

Let's only consider those in the extension of $\mathsf{G}$: Juliet and Beatrice.

When $\mathsf{a}$ is Juliet, $\mathsf{Ga}$ is true, and $\mathsf{Lam \vee \neg Lma}$ is true because $\mathsf{Lam}$ is true, since $\langle \text{Juliet, Romeo}\rangle$ is in the extension of L. So $\mathsf{Gx \to (Lam \vee \neg Lma)}$ is true.

When $\mathsf{a}$ is Beatrice, $\mathsf{Ga}$ is true, and $\mathsf{Lam \vee \neg Lma}$ is true, because $\mathsf{\neg Lma}$ is true, since $\langle \text{Romeo, Beatrice}\rangle$ is not in the extension of L. So $\mathsf{Gx \to (Lam \vee \neg Lma)}$ is true.

By (Q5) $\mathsf{\forall x(Gx \to (Lxm \vee \neg Lmx))}$ is true.

(6) $\mathsf{\forall x(Gx \to \exists yLxy)}$

True.

Let's take the shortcuts again. We need to check the antecedent $\mathsf{\exists y Lay}$ only for the two cases when $\mathsf{Ga}$ is true: Juliet and Beatrice.

When $\mathsf{a}$ is Juliet, $\mathsf{Ga}$ is true, and $\mathsf{\exists y Lay}$ is true, because $\langle \text{Juliet, Romeo}\rangle$ is in the extension of $L$ (informally $\mathsf{y} = $ Romeo).

(7) $\mathsf{\exists x (Fx \wedge \forall y(Gy \to Lxy))}$

{\bf(b)} Now take the following q-valuation:

The domain is: $\{\text{4, 7, 8, 11, 12}\}$

$\mathsf{m}$: 7

$\mathsf{n}$: 12

$\mathsf{F}$: the set of even numbers in the domain

$\mathsf{G}$: the set of odd numbers in the domain

$\mathsf{L}$: the set of pairs $\langle m, n\rangle$ where $m$ and $n$ are in the domain and $m < n$

What are the truth values of the wffs (1) to (7) now?

(1) $\mathsf{\exists xLmx}$

(2) $\mathsf{\forall xLxm}$

(3) $(\mathsf{\exists xLmx\to Lmn})$

(4) $\mathsf{\forall x(Fx \to \neg Gx)}$

(5) $\mathsf{\forall x(Gx \to (Lxm \vee \neg Lmx))}$

(6) $\mathsf{\forall x(Gx \to \exists yLxy)}$

(7) $\mathsf{\exists x (Fx \wedge \forall y(Gy \to Lxy))}$

{\bf(c)} Take the language $\mathsf{QL_3}$ of Exercises 30(b) (copied from that exercise):

This is the language $\mathsf{QL_3}$ whose quantifiers range over the positive integers, with the following glossary:

$\mathsf{n}$: one

$\mathsf{F}$: $\textcircled{\small{1}}$ is odd

$\mathsf{G}$: $\textcircled{\small{1}}$ is even

$\mathsf{H}$: $\textcircled{\small{1}}$ is prime

$\mathsf{L}$: $\textcircled{\small{1}}$ is less than $\textcircled{\small{2}}$

$\mathsf{R}$: $\textcircled{\small{1}}$ is the sum of $\textcircled{\small{2}}$ and $\textcircled{\small{3}}$

Consider the following q-valuation (the natural one suggested by the given glossary for the language):

The domain is the set of natural numbers (the integers from zero up)

$\mathsf{n}$: one

$\mathsf{F}$: the set of odd numbers

$\mathsf{G}$: the set of even numbers

$\mathsf{H}$: the set of prime numbers

$\mathsf{L}$: the set of pairs $\langle m, n\rangle$ such that $m < n$

$\mathsf{R}$: the set of triples $\langle l, m, n\rangle$ such that $l = m + n$.

Carefully work out the values of the wffs (1) to (8) from Exercises 30(b) (copied from that exercise):

(1) $\mathsf{\forall x \forall y \exists z Rzxy}$

(2) $\mathsf{\exists y \forall x Lxy}$

(3) $\mathsf{\forall x \exists y (Lxy \wedge Hy)}$

(4) $\mathsf{\forall x(Hx \to \exists y (Lxy \wedge Hy))}$

(5) $\mathsf{\forall x \forall y ((Fx \wedge Ryxn) \to \neg Fy)}$

(6) $\mathsf{\forall x \exists y ((Gx \wedge Fy) \wedge Rxyy)}$

(7) $\mathsf{\forall x \forall y (\exists z(Rzxn \wedge Ryzn) \to (Gx \to Gy))}$

(8) $\mathsf{\forall x \forall y \forall z (((Fx \wedge Fy) \wedge Rzxy) \to Gz)}$

(9) $\mathsf{\forall x ((Gx \wedge \neg Rxnn) \to \exists y \exists z((Hy \wedge Hz) \wedge Rxyz))}$

(10) $\mathsf{\forall x \exists y ((Hy \wedge Lxy) \wedge \exists w \exists z((Rwyn \wedge Rzwn) \wedge Hz))}$

{\bf(d*)} Show that if the wff $\alpha$ doesn’t contain the dummy name $\delta$, then $\alpha$ is true on the valuation $q$ if and only if it is also true on any expansion $q_\delta$.

No solution available for this one.
\end{document}